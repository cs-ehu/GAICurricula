\href{https://github.com/cs-ehu/GAICurricula/releases}{\texttt{ }}

\subsection*{Índice}


\begin{DoxyItemize}
\item \href{\#gaicurricula}{\texttt{ G\+A\+I\+Curricula}}
\begin{DoxyItemize}
\item \href{\#Índice}{\texttt{ Índice}}
\item \href{\#1-introducción}{\texttt{ 1. Introducción}}
\item \href{\#2-requisitos-para-la-aplicación}{\texttt{ 2. Requisitos para la aplicación}}
\item \href{\#3-instalación-y-uso-de-la-aplicación}{\texttt{ 3. Instalación y uso de la aplicación}}
\begin{DoxyItemize}
\item \href{\#instalación-y-otras-utilidades-a-comentar}{\texttt{ Instalación y otras utilidades a comentar}}
\item \href{\#uso-de-la-aplicación}{\texttt{ Uso de la aplicación}}
\end{DoxyItemize}
\item \href{\#4-estructura-del-proyecto}{\texttt{ 4. Estructura del proyecto}}
\item \href{\#5-licencia}{\texttt{ 5. Licencia}}
\item \href{\#6-autoría}{\texttt{ 6. Autoría}}
\end{DoxyItemize}
\end{DoxyItemize}

\subsection*{1. Introducción}

G\+A\+I\+Curricula es un proyecto que nació en una asignatura llamada Gestión Avanzada de la Información, en la cual la mayor parte de su tiempo se dedica a estudiar tecnologías para X\+ML.

Entre ellas, una de las claves fue X\+Query, tecnología que debía ser parte de un entregable del mismo proyecto. Este era un entregable básico que unicamente consistía en a través de un solo fichero java, generar ficheros H\+T\+ML con información de X\+ML sobre curriculums. A partir de aquí y ya con lo básico, se tuvo la idea de extenderlo y adaptar todo esto a M\+A\+V\+EN.

Su extensión fue a raíz de creer que había potencial para hacer algo más complejo con estas herramientas, dando así como fruto Pom\+H\+T\+ML, una aplicación de escritorio que permite crear un H\+T\+ML básico con toda la información de un pom de un proyecto Java. En pasos posteriores explicaremos como es su instalación y su uso.

\subsection*{2. Requisitos para la aplicación}

Para esta aplicación se necesita unicámente tener {\bfseries{J\+DK 1.\+8}} y {\bfseries{M\+A\+V\+EN}} instalados.

\subsection*{3. Instalación y uso de la aplicación}

\subsubsection*{Instalación y otras utilidades a comentar}

La instalación del proyecto es sencilla, tan solo hace falta una vez importado el proyecto realizar la siguiente instrucción en una ventana de comandos. 
\begin{DoxyCode}{0}
\DoxyCodeLine{mvn clean package assembly:single}
\end{DoxyCode}
 Esta instrucción nos limpiará lo que pudimos hacer anteriormente en el proyecto con instrucciones M\+A\+V\+EN y nos realizará lo que antes hacía en la fase package, pero con la diferencia de que también nos creará el jar ejecutable para utilizar la aplicación. El ejecutable y lo generado estará en la carpeta {\itshape target}, en el caso de que se quiera desinstalar la aplicación será tan fácil como escribir lo siguiente. 
\begin{DoxyCode}{0}
\DoxyCodeLine{mvn clean}
\end{DoxyCode}
 En caso de no querer instalarla, puede encontrar una versión portable \href{https://github.com/cs-ehu/GAICurricula/releases/tag/1.0.1}{\texttt{ aquí}}. \subsubsection*{Uso de la aplicación}

Después de comentar lo referente a la instalación de la aplicación, toca su ejecución. El uso de la aplicación es muy sencillo y lo explicaremos a continuación en unos cuantos pasos.


\begin{DoxyEnumerate}
\item Después de instalar la aplicación, el siguiente paso es ejecutarla, para ello clicamos dos veces en el jar generado y se nos mostrará la pantalla siguiente, dándonos la bienvenida a la aplicación.
\end{DoxyEnumerate}



Este primer punto que nos muestra es importante, ya que debido a un problema por resolver de esta fase temprana del proyecto se pide al usuario que de su fichero pom.\+xml le borre de la etiqueta project todos sus atributos. Una vez se haga esto y aceptemos la notificación, pasamos al segundo paso.




\begin{DoxyEnumerate}
\item Teniendo ya la aplicación lista para usarse, el siguiente paso será examinar y referenciar donde se encuentra nuestro pom, examinar y referenciar donde guardar el fichero H\+T\+ML y por último dar nombre al fichero H\+T\+ML que se va a generar.
\end{DoxyEnumerate}





Aquí hemos dejado un campo sin rellenar para comprobar la notificación que nos saldría si por algún casual nos hemos dejado información, en el caso de rellenar mal estos campos y darle al botón \char`\"{}\+Procesar P\+O\+M\char`\"{} también nos elevará una advertencia para qhe rellenemos correctamente los campos, la aplicación ejecuta todo tipo de comprobaciones para ejecutarse correctamente.


\begin{DoxyEnumerate}
\item Una vez rellenemos correctamente los campos de la aplicación sin dejarnos ningún campo en blanco, ejecutaremos de nuevo.
\end{DoxyEnumerate}



Después de ejecutar nos abrirá la siguiente ventana en caso de que queramos ver el H\+T\+ML generado.



Para este caso nos interesa, así que le daremos que sí y a través de la aplicación que nosotros hayamos definido para abrir los ficheros H\+T\+ML, se ejecutará mostrándonos el fichero, en nuestro caso pusimos el navegador para verlo de una manera gráfica. El fichero H\+T\+ML generado por la aplicación siempre será de este estilo.



Y con todo esto dicho, así es el uso de está aplicación.

\subsection*{4. Estructura del proyecto}

El árbol de estructura general de este proyecto es la siguiente\+:


\begin{DoxyCode}{0}
\DoxyCodeLine{Listado de rutas de carpetas}
\DoxyCodeLine{C:\(\backslash\)Users\(\backslash\)Michel\(\backslash\)workspace\(\backslash\)GAICurricula}
\DoxyCodeLine{|-- doxygen}
\DoxyCodeLine{|   |-- Doxyfile}
\DoxyCodeLine{|   `-- html}
\DoxyCodeLine{|-- imgs}
\DoxyCodeLine{|   |-- app.png}
\DoxyCodeLine{|   |-- fransua.jpg}
\DoxyCodeLine{|   |-- george.jpg}
\DoxyCodeLine{|   |-- hector.jpg}
\DoxyCodeLine{|   `-- john.jpg}
\DoxyCodeLine{|-- input}
\DoxyCodeLine{|   |-- curriculum.xml}
\DoxyCodeLine{|   |-- curriculumEnHTML.xquery}
\DoxyCodeLine{|   |-- curriculumEngland.xml}
\DoxyCodeLine{|   |-- curriculumEnglandSimp.xml}
\DoxyCodeLine{|   |-- curriculumFrHTML.xquery}
\DoxyCodeLine{|   |-- curriculumFrance.xml}
\DoxyCodeLine{|   |-- curriculumFranceSimp.xml}
\DoxyCodeLine{|   |-- erasmus.sch}
\DoxyCodeLine{|   |-- erasmus.xsd}
\DoxyCodeLine{|   |-- francia.sch}
\DoxyCodeLine{|   |-- francia.xsd}
\DoxyCodeLine{|   |-- inglaterra.xsd}
\DoxyCodeLine{|   |-- pom.xml}
\DoxyCodeLine{|   `-- pomHTML.xquery}
\DoxyCodeLine{|-- libraries}
\DoxyCodeLine{|   `-- saxon9-xqj.jar}
\DoxyCodeLine{|-- output}
\DoxyCodeLine{|   |-- curriculumEngland.html}
\DoxyCodeLine{|   |-- curriculumFrance.html}
\DoxyCodeLine{|   `-- pomGAICurricula.html}
\DoxyCodeLine{|-- pom.xml}
\DoxyCodeLine{|-- readme.md}
\DoxyCodeLine{|-- src}
\DoxyCodeLine{|   |-- main}
\DoxyCodeLine{|   |   `-- java}
\DoxyCodeLine{|   |       `-- edu}
\DoxyCodeLine{|   |           `-- ehu}
\DoxyCodeLine{|   |               `-- CS19}
\DoxyCodeLine{|   |                   `-- GAICurricula}
\DoxyCodeLine{|   |                       |-- Data2html.java}
\DoxyCodeLine{|   |                       |-- Pomexec.java}
\DoxyCodeLine{|   |                       `-- XQueryMethods.java}
\DoxyCodeLine{|   `-- test}
\DoxyCodeLine{|       `-- java}
\DoxyCodeLine{|           `-- edu}
\DoxyCodeLine{|               `-- ehu}
\DoxyCodeLine{|                   `-- CS19}
\DoxyCodeLine{|                       `-- GAICurricula}
\DoxyCodeLine{|                           `-- XQueryMethodsTest.java}
\DoxyCodeLine{`-- target}
\DoxyCodeLine{    |-- classes}
\DoxyCodeLine{    |   `-- edu}
\DoxyCodeLine{    |       `-- ehu}
\DoxyCodeLine{    |           `-- CS19}
\DoxyCodeLine{    |               `-- GAICurricula}
\DoxyCodeLine{    |                   |-- Data2html.class}
\DoxyCodeLine{    |                   |-- Pomexec\$1.class}
\DoxyCodeLine{    |                   |-- Pomexec\$2.class}
\DoxyCodeLine{    |                   |-- Pomexec\$3.class}
\DoxyCodeLine{    |                   |-- Pomexec\$4.class}
\DoxyCodeLine{    |                   |-- Pomexec\$5.class}
\DoxyCodeLine{    |                   |-- Pomexec\$6.class}
\DoxyCodeLine{    |                   |-- Pomexec.class}
\DoxyCodeLine{    |                   `-- XQueryMethods.class}
\DoxyCodeLine{    |-- generated-sources}
\DoxyCodeLine{    |   `-- annotations}
\DoxyCodeLine{    `-- test-classes}
\DoxyCodeLine{        `-- edu}
\DoxyCodeLine{            `-- ehu}
\DoxyCodeLine{                `-- CS19}
\DoxyCodeLine{                    `-- GAICurricula}
\DoxyCodeLine{                        `-- XQueryMethodsTest.class}
\DoxyCodeLine{}
\DoxyCodeLine{31 directories, 41 files}
\end{DoxyCode}


\subsection*{5. Licencia}

La licencia utilizada para este proyecto es la {\bfseries{G\+NU General Public License v3.\+0}}. Esta licencia permite en la aplicación su uso comercial, su modificación, su distribución, su uso privado y el poder ser patentada. Para más información sobre la licencia de este proyecto, haga click en \href{https://github.com/FosterGun/GAICurricula/blob/FosterGun/LICENSE}{\texttt{ L\+I\+C\+E\+N\+SE}}.

\subsection*{6. Autoría}

Toda la autoría de este proyecto corre a manos de \href{https://github.com/FosterGun}{\texttt{ Foster\+Gun}}. Si se desea contactar con él puede hacerlo a través de Git\+Hub o a través de su \href{mailto:mblanco040@ikasle.ehu.es}{\texttt{ correo eléctronico}}. 